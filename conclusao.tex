\chapter*{Conclusão}

Neste trabalho, foram apresentadas algumas das principais ameaças que assolam a \textit{Internet}, nos dias de hoje, enfatizando a importância dos ataques \textit{Denial of Service}, devido seu alto grau de periculosidade. Foram também caracterizadas como ameaças relevantes, as redes de pedofilia, organizações criminosas que utilizam os recursos tecnológicos das redes de computadores para a troca de material ilegal entre seus componentes. Como grande problema aos usuários finais de computadores, foram abordados os \textit{malwares}, cujos tipos e danos que podem causar variam, sendo um dos principais fenômenos digitais já presenciados em escala global atualmente.

Tendo motivação para se proteger das ameaças e, principalmente, para estudá-las, foram apresentados diversos conceitos relacionados às tecnologias em \textit{Honeypots}. Foram discutidos os tipos de \textit{Honeypots}, seus propósitos e aspectos legais envolvidos. Aos que se interessarem sobre a implantação de \textit{Honeypots} em suas redes, foram referenciados alguns projetos com reconhecimento mundial, podendo estes serem tomados como ponto de partida.

Sua utilização é de suma importância em grandes organizações, como Universidades e Empresas. Com sua implantação nestes ambientes, pode-se realizar um levantamento extensivo sobre o comportamento das redes, sob condições normais e de risco.

Os \textit{Honeypots} são ferramentas excepcionais para os especialistas em segurança. Revelam novos ataques, assim que estes são lançados, dessa maneira pode-se preparar soluções para os problemas que possam afetar massivamente os sistemas implantados no mundo todo, assim como aumentar o nível de proteção das instituições contra as investidas de atacantes. Avaliam quais são as ameaças recorrentes, auxiliando na melhoria da segurança nas redes e por consequência seus usuários são também protegidos com tais medidas. Inclusive, ajudam a combater diversas organizações criminosas no mundo todo. Sua implantação é recomendada ao time de administradores das organizações, por se tratarem de valiosas ferramentas a serem empregadas na melhoria contínua de seus processos de segurança.

Como trabalho futuro, pode-se realizar um conjunto de instruções completas para a implantação de tecnologias em \textit{Honeypots} em instituições interessadas, fornecendo diretrizes para que tanto a obtenção quanto a análise dos dados possam fazer parte das políticas de segurança empregadas. É de interesse comum, que se realize um estudo de caso no qual as análises realizadas sejam expostas, para que seja validada efetivamente a escolha das tecnologias de acordo com o ambiente em que podem ser inseridas.

