\chapter*{Introdução}

O uso da Internet teve um grande crescimento nos últimos anos \cite{InternetGrowth} e a maioria dos seus usuários são pessoas com baixo nível técnico computacional. É comum que tenham seus computadores infectados por pragas virtuais, que muitas vezes são programadas para lançar ataques em servidores alvo.

Podemos observar uma tendência de aumento na quantidade de dispositivos móveis interligados por meio de redes IP como: telefones celulares, smartphones e sensores \cite{WirelessSensors}. Esse crescimento faz parte de um fenômeno que chamamos de Convergência. Isso nos remete imediatamente à necessidade de que sejam aperfeiçoadas as técnicas de prevenção e detecção de intrusões.

Neste cenário, os profissionais de segurança utilizam Honeypots \cite{UsingHoneypots} como ferramentas de auxílio, sendo máquinas configuradas para coletarem dados sobre ataques sofridos. Seu intuito é atrair atacantes oferecendo um ambiente repleto de falhas de segurança, sem que eles saibam que tais ambientes são controlados e totalmente monitorados.

O sucesso nos ataques irá depender da efetividade das técnicas empregadas e da habilidade de quem os realiza, que por sua vez terá todos os seus passos registrados para uma análise posterior. Essa análise pode revelar um conjunto de informações valiosas que auxiliam a tarefa de manter as redes seguras. Os Honeypots registram atividades legitimamente maliciosas, pois suas identidades como ferramentas de segurança são mantidas sob sigilo, forjando-se comportamentos semelhantes aos apresentados por máquinas reais na rede monitorada.

Com um estudo sobre a natureza e a implantação de \textit{Honeypots}, espera-se gerar um material capaz de agregar conhecimento diferenciado sobre segurança, servindo como base para que os administradores de rede possam estabelecer melhores políticas de segurança em suas instituições. Este trabalho conta com uma descrição em detalhes de cada ítem contido em uma lista de 6 soluções em \textit{Honeypots}. A partir desta iniciativa, espera-se que as tecnologias abordadas possam ser comparadas e o processo de escolha entre as mesmas seja melhor direcionado, visando o sucesso em sua implantação e na identificação/monitoramento de atividades ilegais.

A organização do trabalho obedece a seguinte estrutura:

\begin{itemize}
    \item No capítulo \ref{capitulo:ameacas} serão discutidas quais são, atualmente, as maiores formas de ameaças na \textit{Internet}, mostrando em qual contexto cada uma encontra-se inserida.
    \item No capítulo \ref{capitulo:honeypots} será apresentado o conceito de Honeypot, como forma de auxílio na melhora dos processos de segurança envolvidos nas instituições. Serão abordados os tipos de \textit{Honeypots} encontrados no mercado, boas práticas de implantação, as principais organizações mundiais comprometidas no seu desenvolvimento e as implicações legais pertinentes a sua utilização.
    \item No capítulo \ref{capitulo:solucoes} são apresentadas 6 tipos de soluções compreendidas pelas tecnologias em \textit{Honeypots}. Suas características mais marcantes serão exploradas, no intuito de trazer um apanhado de informções que sejam capazes de guiar a escolha de quais tecnologias serão mais bem aceitas nos ambientes de pesquisa/produção, de acordo com seus propósitos.
\end{itemize}
