\begin{resumo}

\doublespacing {

    \textit{Honeypots} são tecnologias empregadas na indústria de segurança de redes. Seus valores consistem, ao contrário de muitas ferramentas de segurança, em serem reconhecidos, atacados e, eventualmente, comprometidos. Em geral, são constituídos por computadores especialmente preparados para que tais ataques sejam efetivados contra os mesmos, funcionando como isca para diversos tipos de atacantes. \cite{TrackingHackers}

    Ao se estudar os métodos empregados nas investidas contra os \textit{Honeypots}, são geradas várias formas de conhecimento, benéficas no sentido de que podem ser utilizadas para aumentar a segurança dos ambientes de produção e gerar alertas sobre novas ameaças para a comunidade de especialistas em segurança no mundo todo.

    O propósito deste documento é mostrar ao leitor os conceitos envolvidos na temática dos  \textit{Honeypots}. Serão abordadas as vantagens e desvantagens envolvidas em sua implantação, assim como os aspectos legais em sua utilização. Algumas soluções em \textit{Honeypots} serão descritas, tendo suas características mais marcantes expostas, no intuito de incentivar a implantação daquelas que sejam interessantes em cenários compatíveis com seu nicho de atuação.

    \textbf {
        Palavras-chave: Honeypots, Segurança, Redes de Computadores, Hacking.
    }
}

\end{resumo}